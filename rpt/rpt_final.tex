\documentclass[a4paper,10pt]{report}
\usepackage{graphicx}
\usepackage[polish]{babel}
\usepackage[utf8]{inputenc}
\usepackage{polski}
\usepackage[T1]{fontenc}
\usepackage{indentfirst}
\usepackage{amsmath}
\usepackage{verbatim}
\usepackage{booktabs}
\usepackage{color}
\usepackage[usenames,dvipsnames,svgnames]{xcolor}
\usepackage{geometry}
\geometry{verbose,lmargin=3cm,rmargin=3cm}
\usepackage{utopia}
\frenchspacing

%opening
\title{Rozpoznawanie relacji sematycznych w tekscie, porzez klasyfikację kontekstu relacji przy pomocy klasyfikatora CRF}
\author{Jakub A. Gramsz \\ Michał Krautforst}
\date{24 stycznia 2014}

\begin{document}
\renewcommand{\figurename}{Wykres}
\renewcommand{\chaptername}{}

\maketitle
\tableofcontents

\chapter{Wstęp}

\section{Klasyfikator CRF}

Klasyfikator CRF został zaporoponowany aby przetwaorzać sekwęcje w obu kierunkach \cite{lafferty2001crf}.

\section{Rozpoznanie literatury}

\chapter{Praca badawcza}

\section{Implementacja}

\section{Badania}

\chapter{Wyniki i wnioski}

\section{Wyniki}

\section{Wnioski}

\bibliographystyle{abbrv}
\nocite{*}
\bibliography{sem-rel-crf}

\end{document}
